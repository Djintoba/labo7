\documentclass{beamer}
\usepackage[utf8]{inputenc}
\usepackage{graphicx}
\usepackage{amsmath, amssymb, amsfonts}
\usepackage{multicol}

% Theme choice
\usetheme{Madrid}
\usecolortheme{default}

\title{Mathematical Concepts Presentation}
\author{NONOGA Djintoba}
\institute{RUDN University}
\date{\today}

\begin{document}

% Title page
\begin{frame}
    \titlepage
    \centering
    \includegraphics[width=0.3\textwidth]{logorudn.png}
\end{frame}

% Table of Contents
\begin{frame}{Table of Contents}
    \tableofcontents
\end{frame}

% Introduction with objectives
\section{Introduction \& Objectives}
\begin{frame}{Introduction \& Objectives}
    \begin{columns}
        \begin{column}{0.7\textwidth}
            \textbf{Presentation Goals:}
            \begin{itemize}
                \item Introduce fundamental set theory concepts
                \item Explain common mathematical sets
                \item Demonstrate calculus product rule
                \item Show practical applications
            \end{itemize}
            
            \vspace{0.5cm}
            \textbf{Key Topics:}
            \begin{itemize}
                \item Sets and elements
                \item Special number sets ($\mathbb{N}$, $\mathbb{Z}$, $\mathbb{Q}$, $\mathbb{R}$)
                \item Derivative product rule
            \end{itemize}
        \end{column}
        
        \begin{column}{0.3\textwidth}
            \centering
            \includegraphics[width=\textwidth]{logorudn.png}
            \\
            \vspace{0.5cm}
            \footnotesize
            RUDN University\\
            Moscow, Russia
        \end{column}
    \end{columns}
\end{frame}

% Combined sets and operations
\section{Set Theory Fundamentals}
\begin{frame}{Set Theory Fundamentals}
    \begin{columns}
        \begin{column}{0.48\textwidth}
            \textbf{Basic Definitions:}
            \begin{itemize}
                \item Set: Collection of objects
                \item Example: $Z = \{\text{cow}, \text{pig}, \text{elephant}\}$
                \item Element: $\text{cow} \in Z$
                \item Common sets:
                \begin{itemize}
                    \item $\mathbb{N} = \{1,2,3,\ldots\}$ (Natural)
                    \item $\mathbb{Z} = \{\ldots,-2,-1,0,1,2,\ldots\}$ (Integer)
                    \item $\mathbb{Q} = \{p/q : p,q\in\mathbb{Z}, q\neq 0\}$ (Rational)
                    \item $\mathbb{R}$ = Decimal numbers (Real)
                \end{itemize}
            \end{itemize}
        \end{column}
        
        \begin{column}{0.48\textwidth}
            \textbf{Set Operations:}
            \begin{itemize}
                \item Union: $A \cup B = \{x : x \in A \text{ or } x \in B\}$
                \item Intersection: $A \cap B = \{x : x \in A \text{ and } x \in B\}$
                \item Difference: $A \setminus B = \{x : x \in A \text{ and } x \notin B\}$
            \end{itemize}
            
            \vspace{0.5cm}
            \textbf{Example:}
            If $A = \{1,2,3\}$ and $B = \{2,3,4\}$:
            \begin{itemize}
                \item $A \cup B = \{1,2,3,4\}$
                \item $A \cap B = \{2,3\}$
                \item $A \setminus B = \{1\}$
            \end{itemize}
        \end{column}
    \end{columns}
\end{frame}

% Calculus with proof combined
\section{Calculus: Product Rule}
\begin{frame}{Calculus: Product Rule}
    \begin{columns}
        \begin{column}{0.65\textwidth}
            \textbf{Product Rule Formula:}
            \[
            \frac{d}{dx}[f(x)g(x)] = f'(x)g(x) + f(x)g'(x)
            \]
            
            \textbf{Example:}
            For $f(x) = x^2 \cdot \sin(x)$:
            \begin{align*}
            f'(x) &= \frac{d}{dx}(x^2) \cdot \sin(x) + x^2 \cdot \frac{d}{dx}(\sin(x)) \\
            &= 2x \cdot \sin(x) + x^2 \cdot \cos(x)
            \end{align*}
            
            \textbf{Proof Outline:}
            \begin{itemize}
                \item Start with definition: $f'(x) = \lim_{h\to 0} \frac{f(x+h)-f(x)}{h}$
                \item For $f(x) = g(x)h(x)$:
                \item Add and subtract $g(x+h)h(x)$
                \item Result: $f'(x) = g(x)h'(x) + g'(x)h(x)$
            \end{itemize}
        \end{column}
        
        \begin{column}{0.35\textwidth}
            \centering
            \includegraphics[width=0.8\textwidth]{example-image}
            \\
            \footnotesize
            Visual: Product rule
            
            \vspace{1cm}
            \textbf{Key Insight:}
            \small
            The derivative of a product is NOT the product of derivatives!
            \[
            (fg)' \neq f'g'
            \]
        \end{column}
    \end{columns}
\end{frame}

% Applications and conclusion combined
\section{Applications \& Conclusion}
\begin{frame}{Applications \& Conclusion}
    \begin{columns}
        \begin{column}{0.7\textwidth}
            \textbf{Applications:}
            \begin{itemize}
                \item \textbf{Physics:} Motion equations, force calculations
                \item \textbf{Engineering:} Signal processing, control systems
                \item \textbf{Computer Science:} Algorithms, machine learning
                \item \textbf{Economics:} Optimization, modeling
            \end{itemize}
            
            \vspace{0.5cm}
            \textbf{Key Takeaways:}
            \begin{enumerate}
                \item Sets provide foundation for mathematical structures
                \item Product rule is essential for calculus operations
                \item These concepts enable real-world problem solving
            \end{enumerate}
        \end{column}
        
        \begin{column}{0.3\textwidth}
            \centering
            \includegraphics[width=0.9\textwidth]{logorudn.png}
            
            \vspace{0.5cm}
            \textbf{Thank You!}
            
            \vspace{0.2cm}
            \small
            Questions?
            
            \vspace{0.5cm}
            \footnotesize
            NONOGA Djintoba\\
            RUDN University
        \end{column}
    \end{columns}
    
    \vspace{0.5cm}
    \centering
    \rule{0.8\textwidth}{0.5pt}
    
    \small
    \textbf{References:}
    Calculus by James Stewart, Discrete Mathematics by Rosen
\end{frame}

\end{document}